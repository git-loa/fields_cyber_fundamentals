
\documentclass[12pt]{ut-thesis}
\usepackage[T1]{fontenc}
\usepackage[utf8]{inputenc}
%\usepackage{mathptmx}
%\usepackage{apacite}
%\usepackage{cite}
\usepackage[numbers]{natbib}
\usepackage{enumerate}
\usepackage{enumitem}
\usepackage{graphicx}
\usepackage[usenames,dvipsnames]{color}
\usepackage{listings}
%\usepackage{diagbox}
\usepackage{rotating}
\usepackage{chngcntr}
\usepackage{multicol}
\usepackage[bf,small,tableposition=top]{caption}
\usepackage{subcaption}

\usepackage{titlesec}
\usepackage{amssymb}
\usepackage{amsmath}
\usepackage{amsthm}
\usepackage{mathdots}
\usepackage{amsfonts}
\usepackage{tcolorbox}
%\usepackage{extarrows}
%\usepackage{moreverb}
\usepackage[page]{totalcount}
\usepackage{totcount}
\regtotcounter{page}
\usepackage[hang, flushmargin]{footmisc}
\usepackage[colorlinks=true]{hyperref}
%\usepackage[pdftex, hyperfootnotes=false, colorlinks=false]{hyperref}
%\usepackage{footnotebackref}


\usepackage{tikz-cd}

\usepackage{fancyhdr}
\setlength{\headheight}{14.5pt}

\pagestyle{fancy}
\renewcommand{\headrulewidth}{0.4pt}
\renewcommand{\footrulewidth}{0.4pt}


\fancyhf{}
\fancyfoot[R]{(\thepage~ of \total{page})}
\lfoot{Notes and solutions by Leonard O. Afeke \today}
\chead{Cyber Security Fundamentals | Mathematics (Exercise 3.1)}

\DeclareMathOperator{\Frob}{Frob}
\DeclareMathOperator{\plim}{\underleftarrow{\lim}^{i}}
\DeclareMathOperator{\lm}{\underleftarrow{\lim}^{*}}

\parskip=0.5\baselineskip
\parindent=0pt
\renewcommand{\baselinestretch}{1.5}

%%%%%%%%%%%% Line spacesing codes %%%%%%%%%%%%
\usepackage{setspace}
%\singlespacing
\onehalfspacing
%\doublespacing
%\setstretch{1.1}

\usepackage{lineno} % This package together with lineno.sty numbers every line. Makes it easy for edditing.

\newtheorem{thm}[subsection]{Theorem}
% Rest is not in italics.
%\theoremstyle{definition}
\newtheorem{lem}[subsection]{Lemma}
\newtheorem{cor}[subsection]{Corollary}
\newtheorem{conj}[subsection]{Conjecture}
\newtheorem{pro}[subsection]{Proposition}

\theoremstyle{definition}
\newtheorem{defn}[subsection]{Definition}
\newtheorem{rem}[subsection]{Remark}
\newtheorem{exa}[subsection]{Example}
\newtheorem{con}[subsection]{Condition}

\newcommand{\ora}[1]{{\xrightarrow{\hspace{0.2cm #1 \hspace{0.2cm}}}}}
\newcommand{\ola}[1]{{\xleftarrow{#1}}}
\newcommand{\hm}[3]{\mathrm{Hom}_{\mathcal{#1}}({#2}, {#3})}
\newcommand{\dor}[1]{{\xrightarrow{#1}}}
\newcommand{\sol}[1]{{\bf \emph{Solution}} #1}

\newcommand{\cA}{\mathcal{A}}\newcommand{\cB}{\mathcal{B}}
\newcommand{\cC}{\mathcal{C}}\newcommand{\cD}{\mathcal{D}}
\newcommand{\cE}{\mathcal{E}}\newcommand{\cF}{\mathcal{F}}
\newcommand{\cG}{\mathcal{G}}\newcommand{\cH}{\mathcal{H}}
\newcommand{\cI}{\mathcal{I}}\newcommand{\cJ}{\mathcal{J}}
\newcommand{\cK}{\mathcal{K}}\newcommand{\cL}{\mathcal{L}}
\newcommand{\cM}{\mathcal{M}}\newcommand{\cN}{\mathcal{N}}
\newcommand{\cO}{\mathcal{O}}\newcommand{\cP}{\mathcal{P}}
\newcommand{\cQ}{\mathcal{Q}}\newcommand{\cR}{\mathcal{R}}
\newcommand{\cS}{\mathcal{S}}\newcommand{\cT}{\mathcal{T}}
\newcommand{\cU}{\mathcal{U}}\newcommand{\cV}{\mathcal{V}}
\newcommand{\cW}{\mathcal{W}}\newcommand{\cX}{\mathcal{X}}
\newcommand{\cY}{\mathcal{Y}}\newcommand{\cZ}{\mathcal{Z}}

\newcommand{\bA}{\mathbb{A}}\newcommand{\bB}{\mathbb{B}}
\newcommand{\bC}{\mathbb{C}}\newcommand{\bD}{\mathbb{D}}
\newcommand{\bE}{\mathbb{E}}\newcommand{\bF}{\mathbb{F}}
\newcommand{\bG}{\mathbb{G}}\newcommand{\bH}{\mathbb{H}}
\newcommand{\bI}{\mathbb{I}}\newcommand{\bJ}{\mathbb{J}}
\newcommand{\bK}{\mathbb{K}}\newcommand{\bL}{\mathbb{L}}
\newcommand{\bM}{\mathbb{M}}\newcommand{\bN}{\mathbb{N}}
\newcommand{\bO}{\mathbb{O}}\newcommand{\bP}{\mathbb{P}}
\newcommand{\bQ}{\mathbb{Q}}\newcommand{\RR}{\mathbb{R}}
\newcommand{\bS}{\mathbb{S}}\newcommand{\bT}{\mathbb{T}}
\newcommand{\bU}{\mathbb{U}}\newcommand{\bV}{\mathbb{V}}
\newcommand{\bW}{\mathbb{W}}\newcommand{\bX}{\mathbb{X}}
\newcommand{\bY}{\mathbb{Y}}\newcommand{\bZ}{\mathbb{Z}}


\newcommand{\B}[1]{\mathbf{#1}}
\newcommand{\BB}[1]{\mathbb{#1}}
\definecolor{codegreen}{rgb}{0,0.6,0}
\definecolor{codegray}{rgb}{0.5,0.5,0.5}
\definecolor{codepurple}{rgb}{0.58,0,0.82}
\definecolor{backcolour}{rgb}{0.95,0.95,0.92}

\lstdefinestyle{mystyle}{
    backgroundcolor=\color{backcolour},   
    commentstyle=\color{codegreen},
    keywordstyle=\color{magenta},
    numberstyle=\tiny\color{codegray},
    stringstyle=\color{codepurple},
    basicstyle=\ttfamily\footnotesize,
    breakatwhitespace=false,         
    breaklines=true,                 
    captionpos=b,                    
    keepspaces=true,                 
    numbers=left,                    
    numbersep=5pt,                  
    showspaces=false,                
    showstringspaces=false,
    showtabs=false,                  
    tabsize=2
}
\lstset{style=mystyle}
\begin{document}
	\section*{Exercise 1.45}
    Let $N$ be a large integer and let $K = M = C = \mathbb{Z}/N\mathbb{Z}$. 
    For each of the functions $$e : K \times M \rightarrow C$$
    listed in (a)–(c), answer the following questions:
    \begin{itemize}
        \item[1] Is $e$ an encryption function?
        \item[2.] If $e$ is an encryption function, what is its associated decryption function d?
        \item[3.] If $e$ is not an encryption function, can you make it into an encryption 
        function by using some smaller, yet reasonably large, set of keys?    
    \end{itemize}
    \begin{itemize}
        \item[(a)] $e_k(m) \equiv k - m(\mod N)$
        
        \textbf{Ans:}
        \begin{enumerate}
            \item Yes, $e_k(m) \equiv k - m\pmod N$ is an encryption function, 
            since subtraction is reversible. 
            \item To recover the message $m$ from the ciphertext $c$, we can solve for $m$ using
            $m \equiv k - c \pmod{N}$. Thus, the decryption function is the function $d:K\times C \rightarrow M$
            defined by  $$d_k(c) \equiv k - c \pmod{N}$$ 
        \end{enumerate}
        \item[(b)] $e_k(m) \equiv k \cdot m\pmod N$
        
        \textbf{Ans:}
        \begin{enumerate}
            \item It depends on whether or not $k$ has a modular inverse. 
            If $k$ has a modular inverse, then  $e_k(m) \equiv k\cdot m\pmod N$ is an encryption function. 
            Here, $k$ has modular inverse if $(k, N) = 1$.
            \item  If $k$ is invertible modulo $N$, define its modular inverse 
            $k^{-1}$, such that $k^{-1} \cdot k \equiv 1 \pmod{N}$. Then the decryption function is  
            $$d_k(c) \equiv k^{-1} \cdot c \pmod{N}.$$
        \end{enumerate}
        \item[(c)] $e_k(m) \equiv (k + m)^2\pmod N$
        
        \textbf{Ans:}
        \begin{enumerate}
            \item No, this function is not always invertible, because squaring loses 
            sign information and can lead to collisions (that is, multiple plaintexts mapping to the same ciphertext).

            \item  If $k + m$ is restricted to a subset of values that allow unique reversibility, then we can
            make the function an encryption function. For example, working only within a specific modular residue class
            such as the subset of quadratic residues modulo $N$.


        \end{enumerate}
    \end{itemize}
    
    
    


    \section*{Exercise 1.47}
        Alice and Bob choose a key space $K$ containing 256 keys. Eve builds a special-purpose 
        computer that can check $10,000,000,000$ keys per second.

        \begin{itemize}
            \item[(a)] How many days does it take Eve to check half of the keys in $K$?
            
            \textbf{Ans}

            The time required for Eve to check half the keys in $K$ is computed as follows.
            Half the keys is $\frac{256}{2} = 128 \text{ keys}.$ At a rate of $10^{10}$ keys per second, the time taken for Alice 
            to check half of the keys in K is $\frac{128}{10^{10}} \text{ seconds} = 1.28 \times 10^{-8} \text{ seconds}$ which 
            is approximately $\frac{1.28}{864\times 10^{10}}$ days.


           

            \item[(b)] Alice and Bob replace their key space with a larger set containing $2^B$ different keys. 
            How large should Alice and Bob choose $B$ in order to force Eve's computer to spend $100$ years 
            checking half the keys? (Use the approximation that there are $365.25$ days in a year.)

            \textbf{Ans}

            Half the keys is $\frac{2^B}{2} = 2^{B-1}$, and 100 years in seconds is 
            $100 \times 365.25 \times 24 \times 60 \times 60 =3155760000 \text{ seconds}$. Set up the equation
            $$\frac{2^{B-1}}{10^{10}} \text{ sec} = 3155760000 \text{ sec},$$ to obtain 
            $$2^{B-1}= 3.15576\times 10^{19}.$$
            Take logarithms to get
            \begin{eqnarray*}
                B &=& 1 + \log_2(3.15576\times 10^{19})\\
                 &=& 1 + \frac{\log_{10}(3.15576) + \log_{10}(10^{19})}{\log_{10}(2)}\\
                 &=& 1 + \frac{\log_{10}(3.15576) + 19}{\log_{10}(2)}\\
                 &=& 1 + \frac{19.4991}{0.3010}\\
                 &=& 65.7746.
            \end{eqnarray*}
            Thus, Alice and Bob should choose $B \approx 66$ to ensure 100 years of security against Eve's computer.


        \end{itemize}  
        
        
\end{document}
