\documentclass[12pt]{ut-thesis}
\usepackage[T1]{fontenc}
\usepackage[utf8]{inputenc}
%\usepackage{mathptmx}
%\usepackage{apacite}
%\usepackage{cite}
\usepackage[numbers]{natbib}
\usepackage{enumerate}
\usepackage{enumitem}
\usepackage{graphicx}
\usepackage[usenames,dvipsnames]{color}
\usepackage{listings}
%\usepackage{diagbox}
\usepackage{rotating}
\usepackage{chngcntr}
\usepackage{multicol}
\usepackage[bf,small,tableposition=top]{caption}
\usepackage{subcaption}

\usepackage{titlesec}
\usepackage{amssymb}
\usepackage{amsmath}
\usepackage{amsthm}
\usepackage{mathdots}
\usepackage{amsfonts}
\usepackage{tcolorbox}
%\usepackage{extarrows}
%\usepackage{moreverb}
\usepackage[page]{totalcount}
\usepackage{totcount}
\regtotcounter{page}
\usepackage[hang, flushmargin]{footmisc}
\usepackage[colorlinks=true]{hyperref}
%\usepackage[pdftex, hyperfootnotes=false, colorlinks=false]{hyperref}
%\usepackage{footnotebackref}


\usepackage{tikz-cd}

\usepackage{fancyhdr}
\setlength{\headheight}{14.5pt}

\pagestyle{fancy}
\renewcommand{\headrulewidth}{0.4pt}
\renewcommand{\footrulewidth}{0.4pt}


\fancyhf{}
\fancyfoot[R]{(\thepage~ of \total{page})}
\lfoot{Notes and solutions by Leonard O. Afeke \today}
\chead{Cyber Security Fundamentals | Mathematics (Exercise 3.1)}

\DeclareMathOperator{\Frob}{Frob}
\DeclareMathOperator{\plim}{\underleftarrow{\lim}^{i}}
\DeclareMathOperator{\lm}{\underleftarrow{\lim}^{*}}

\parskip=0.5\baselineskip
\parindent=0pt
\renewcommand{\baselinestretch}{1.5}

%%%%%%%%%%%% Line spacesing codes %%%%%%%%%%%%
\usepackage{setspace}
%\singlespacing
\onehalfspacing
%\doublespacing
%\setstretch{1.1}

\usepackage{lineno} % This package together with lineno.sty numbers every line. Makes it easy for edditing.

\newtheorem{thm}[subsection]{Theorem}
% Rest is not in italics.
%\theoremstyle{definition}
\newtheorem{lem}[subsection]{Lemma}
\newtheorem{cor}[subsection]{Corollary}
\newtheorem{conj}[subsection]{Conjecture}
\newtheorem{pro}[subsection]{Proposition}

\theoremstyle{definition}
\newtheorem{defn}[subsection]{Definition}
\newtheorem{rem}[subsection]{Remark}
\newtheorem{exa}[subsection]{Example}
\newtheorem{con}[subsection]{Condition}

\newcommand{\ora}[1]{{\xrightarrow{\hspace{0.2cm #1 \hspace{0.2cm}}}}}
\newcommand{\ola}[1]{{\xleftarrow{#1}}}
\newcommand{\hm}[3]{\mathrm{Hom}_{\mathcal{#1}}({#2}, {#3})}
\newcommand{\dor}[1]{{\xrightarrow{#1}}}
\newcommand{\sol}[1]{{\bf \emph{Solution}} #1}

\newcommand{\cA}{\mathcal{A}}\newcommand{\cB}{\mathcal{B}}
\newcommand{\cC}{\mathcal{C}}\newcommand{\cD}{\mathcal{D}}
\newcommand{\cE}{\mathcal{E}}\newcommand{\cF}{\mathcal{F}}
\newcommand{\cG}{\mathcal{G}}\newcommand{\cH}{\mathcal{H}}
\newcommand{\cI}{\mathcal{I}}\newcommand{\cJ}{\mathcal{J}}
\newcommand{\cK}{\mathcal{K}}\newcommand{\cL}{\mathcal{L}}
\newcommand{\cM}{\mathcal{M}}\newcommand{\cN}{\mathcal{N}}
\newcommand{\cO}{\mathcal{O}}\newcommand{\cP}{\mathcal{P}}
\newcommand{\cQ}{\mathcal{Q}}\newcommand{\cR}{\mathcal{R}}
\newcommand{\cS}{\mathcal{S}}\newcommand{\cT}{\mathcal{T}}
\newcommand{\cU}{\mathcal{U}}\newcommand{\cV}{\mathcal{V}}
\newcommand{\cW}{\mathcal{W}}\newcommand{\cX}{\mathcal{X}}
\newcommand{\cY}{\mathcal{Y}}\newcommand{\cZ}{\mathcal{Z}}

\newcommand{\bA}{\mathbb{A}}\newcommand{\bB}{\mathbb{B}}
\newcommand{\bC}{\mathbb{C}}\newcommand{\bD}{\mathbb{D}}
\newcommand{\bE}{\mathbb{E}}\newcommand{\bF}{\mathbb{F}}
\newcommand{\bG}{\mathbb{G}}\newcommand{\bH}{\mathbb{H}}
\newcommand{\bI}{\mathbb{I}}\newcommand{\bJ}{\mathbb{J}}
\newcommand{\bK}{\mathbb{K}}\newcommand{\bL}{\mathbb{L}}
\newcommand{\bM}{\mathbb{M}}\newcommand{\bN}{\mathbb{N}}
\newcommand{\bO}{\mathbb{O}}\newcommand{\bP}{\mathbb{P}}
\newcommand{\bQ}{\mathbb{Q}}\newcommand{\RR}{\mathbb{R}}
\newcommand{\bS}{\mathbb{S}}\newcommand{\bT}{\mathbb{T}}
\newcommand{\bU}{\mathbb{U}}\newcommand{\bV}{\mathbb{V}}
\newcommand{\bW}{\mathbb{W}}\newcommand{\bX}{\mathbb{X}}
\newcommand{\bY}{\mathbb{Y}}\newcommand{\bZ}{\mathbb{Z}}


\newcommand{\B}[1]{\mathbf{#1}}
\newcommand{\BB}[1]{\mathbb{#1}}
\definecolor{codegreen}{rgb}{0,0.6,0}
\definecolor{codegray}{rgb}{0.5,0.5,0.5}
\definecolor{codepurple}{rgb}{0.58,0,0.82}
\definecolor{backcolour}{rgb}{0.95,0.95,0.92}

\lstdefinestyle{mystyle}{
    backgroundcolor=\color{backcolour},   
    commentstyle=\color{codegreen},
    keywordstyle=\color{magenta},
    numberstyle=\tiny\color{codegray},
    stringstyle=\color{codepurple},
    basicstyle=\ttfamily\footnotesize,
    breakatwhitespace=false,         
    breaklines=true,                 
    captionpos=b,                    
    keepspaces=true,                 
    numbers=left,                    
    numbersep=5pt,                  
    showspaces=false,                
    showstringspaces=false,
    showtabs=false,                  
    tabsize=2
}
\lstset{style=mystyle}
\begin{document}
	\section*{Exercise 5.1 from textbook}
	The Rhind papyrus is an ancient Egyptian mathematical manuscript that is more than 3500 years old.
	Problem 79 of the Rhind papyrus poses a problem that can be paraphrased as follows: there are seven houses;
	in each house lives seven cats; each cat kills seven mice; each mouse has eaten seven spelt seeds; each spelt seed
	would have produced seven hekat of spelt. What is the sum of all of the named
	items? Solve this 3500 year old problem.

	\textbf{Solution}\\
	There are 7 houses. In each house lives 7 cats. This implies that there are $7\times 7 =49$ cats in total.
	Next, each of the 49 cats kills 7 mice. This implies that there are $49 \times 7 = 343$ mice
	in total. Again since each of the 343 mice has eaten 7 spelt seeds this again implies that there are $343 \times 7 = 2401$
	seeds in total. Finally, each spelt seed would have produced 7 hekat, so there are $2401 \times 7 = 16807$ kehat
	in total.

	Now, the sum of all the named items is $7 + 49 + 2401 + 16807 = 19607$.

	\section*{Exercise 5.2 from textbook}
	\begin{itemize}
		\item[(a)] How many n-tuples $(x_1, x_2, \cdots , x_n)$ are there if the coordinates are required
		to be integers satisfying $0 \le x_i < q$?\\
		\textbf{Ans:} There are $q$ possible choices for each $x_i$, since \( 0 \leq x_i < q \). 
		Since there are $n$ independent coordinates, the total number of n-tuples is $q^n$.

		\item[(b)] Same question as (a), except now there are separate bounds $0 \le x_i < q_i$ for
		each coordinate.\\
		\textbf{Ans:} There are $q_i$ choices for each $x_i$. It follows that the total number of 
		n-tuples is $q_1 q_2\cdots q_n$.

		\item[(c)] How many $n\times n$ matrices are there if the entries $x_{i,j}$ of the matrix are integers
		satisfying $0 \le x_{i,j} < q$?\\
		\textbf{Ans:} By part (a), total number of $n \times n$ matrices with entries $x_{i,j}$ are $q^{n^2}$.

		\item[(d)] Same question as (a), except now the order of the coordinates does not matter.
		So for example, (0, 0, 1, 3) and (1, 0, 3, 0) are considered the same. (This one is
		rather tricky.)

		\textbf{Ans:} Since the order of the coordinates does not matter, this problem follows the combinatorial 
		principle of combinations with repetition. So, we can think of it as placing $n$ objects into $q$ containers.
		In other words, this is choosing $n$ items from a total of $n+q-1$ items, which is
		given by $\binom{q + n - 1}{n}$.

		\item[(e)] Twelve students are each taking four classes, for each class they need two looseleaf
		notebooks, for each notebook they need 100 sheets of paper, and each sheet
		of paper has 32 lines on it. Altogether, how many students, classes, notebooks,
		sheets, and lines are there?\\
		\textbf{Ans:}
		Given 12 students, each enrolled in 4 classes, the total class count is $12 \times 4 = 48$. 
		Each class requires 2 notebooks, yielding $48 \times 2 = 96$ notebooks. 
		Each notebook contains 100 sheets, resulting in  $96 \times 100 = 9600$  sheets.
		Each sheet holds 32 lines, leading to $9600 \times 32 = 307,200$ lines.
		The classroom collectively comprises 12 students, 48 classes, 96 notebooks, 9,600 sheets, and 307,200 lines.
		So, the total count of all items considered in the classroom is $316956$.
	\end{itemize}

	\section*{Exercise 5.9 from textbook}
	We know that there are n! different permutations of the set $\{1, 2, \cdots, n\}$
	\begin{itemize}
		\item[(a)] How many of these permutations leave no number fixed?
		\item[(b)] How many of these permutations leave at least one number fixed?
		\item[(c)] How many of these permutations leave exactly one number fixed?
		\item[(d)] How many of these permutations leave at least two numbers fixed?
	\end{itemize}
	For each part of this problem, give a formula or algorithm that can be used to
	compute the answer for an arbitrary value of n, and then compute the value for 
	$n=10$ and $n = 26$.

	\textbf{Answers}
	\begin{itemize}
		\item[(a)]This is the number of derangement. That is, the number of permutations that leave no number fixes. 
		The formula for this permutation, denoted as $!n$ is given by
		\[ !n = n! \sum_{k=0}^{n} \frac{(-1)^k}{k!} .\]
		
		\begin{lstlisting}[language=Python, caption=Exercise 5.9 (a)- Derangement python Implementation]
			import math
			def derangement(n):
				"""Compute the number of derangements (!n) using the inclusion-exclusion principle."""
				return round(
					math.factorial(n) * sum((-1) ** k / math.factorial(k) for k in range(n + 1))
				)
		\end{lstlisting}

		The summation is approximately $\frac{1}{e}$ for large values of $n$. So,
		\begin{itemize}
			\item For $n=10$, we have $!10 = 1334961$ 
			\item For $n=26$, we have $!26 = 148362637348470138328842240$
		\end{itemize}
		\item[(b)] This is the complement of derangements. So, the number of permutations that leave at least 
		one number fixed is given by $n! - !n$.
		\begin{itemize}
			\item For $n=10$, we have $10! - !10 = 3628800 - 1334961 = 2293839$ 
			\item For $n=26$, we have $26! - !26  = 254928823778135497255157760$
		\end{itemize}
		\item[(c)] To count permutations that fix exactly one element and move the others, pick one fixed element from the $n$
		choices and permute the remaining 9 elemnts. This is given by $n \times !(n-1).$
		\begin{itemize}
			\item For $n=10$, we have $10 \times !9 = 10 \times 133496 = 1334960$ 
			\item For $n=26$, we have $26 \times !25 = 148362637348470127591424000$
		\end{itemize}
		\item[(d)] The number of permutations that leave at least two numbers fixed is given by $ n! - n \cdot !(n-1).$
		\item[] \begin{itemize}
			\item For $n=10$, we have $10! - 10 \cdot !(9)  = 2293840$ 
			\item For $n=26$, we have $26! - 26 \cdot !(25) =  254928823778135507992576000$
		\end{itemize}
	\end{itemize}
\end{document}
