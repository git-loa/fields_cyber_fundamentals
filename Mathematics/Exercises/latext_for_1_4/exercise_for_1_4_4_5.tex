\documentclass[12pt]{article}
\usepackage[T1]{fontenc}
\usepackage[utf8]{inputenc}
%\usepackage{mathptmx}
%\usepackage{apacite}
%\usepackage{cite}

\usepackage{geometry}
 \geometry{
 a4paper,
 total={170mm,257mm},
 left=20mm,
 top=20mm,
 }

\usepackage[numbers]{natbib}
\usepackage{enumerate}
\usepackage{enumitem}
\usepackage{graphicx}
\usepackage[usenames,dvipsnames]{color}
\usepackage{listings}

%\usepackage{diagbox}
\usepackage{rotating}
\usepackage{chngcntr}
\usepackage{multicol}
\usepackage[bf,small,tableposition=top]{caption}
\usepackage{subcaption}

\usepackage{titlesec}
\usepackage{amssymb}
\usepackage{amsmath}
\usepackage{amsthm}
\usepackage{mathdots}
\usepackage{amsfonts}
\usepackage{tcolorbox}
%\usepackage{extarrows}
%\usepackage{moreverb}
\usepackage[page]{totalcount}
\usepackage{totcount}
\regtotcounter{page}
\usepackage[hang, flushmargin]{footmisc}
\usepackage[colorlinks=true]{hyperref}
%\usepackage[pdftex, hyperfootnotes=false, colorlinks=false]{hyperref}
%\usepackage{footnotebackref}


\usepackage{tikz-cd}

\usepackage{fancyhdr}
\setlength{\headheight}{14.5pt}

\pagestyle{fancy}
\renewcommand{\headrulewidth}{0.4pt}
\renewcommand{\footrulewidth}{0.4pt}


\fancyhf{}
\fancyfoot[R]{(\thepage~ of \total{page})}
\lfoot{Authored by Leonard O. Afeke \today}
\chead{Cyber security Fundamentals | Mathematics}

\DeclareMathOperator{\Frob}{Frob}
\DeclareMathOperator{\plim}{\underleftarrow{\lim}^{i}}
\DeclareMathOperator{\lm}{\underleftarrow{\lim}^{*}}

\parskip=0.5\baselineskip
\parindent=0pt
\renewcommand{\baselinestretch}{1.5}

%%%%%%%%%%%% Line spacesing codes %%%%%%%%%%%%
\usepackage{setspace}
%\singlespacing
\onehalfspacing
%\doublespacing
%\setstretch{1.1}

\usepackage{lineno} % This package together with lineno.sty numbers every line. Makes it easy for edditing.

\newtheorem{thm}[subsection]{Theorem}
% Rest is not in italics.
%\theoremstyle{definition}
\newtheorem{lem}[subsection]{Lemma}
\newtheorem{cor}[subsection]{Corollary}
\newtheorem{conj}[subsection]{Conjecture}
\newtheorem{pro}[subsection]{Proposition}

\theoremstyle{definition}
\newtheorem{defn}[subsection]{Definition}
\newtheorem{rem}[subsection]{Remark}
\newtheorem{exa}[subsection]{Example}
\newtheorem{con}[subsection]{Condition}

\newcommand{\ora}[1]{{\xrightarrow{\hspace{0.2cm #1 \hspace{0.2cm}}}}}
\newcommand{\ola}[1]{{\xleftarrow{#1}}}
\newcommand{\hm}[3]{\mathrm{Hom}_{\mathcal{#1}}({#2}, {#3})}
\newcommand{\dor}[1]{{\xrightarrow{#1}}}
\newcommand{\sol}[1]{{\bf \emph{Solution}} #1}

\newcommand{\cA}{\mathcal{A}}\newcommand{\cB}{\mathcal{B}}
\newcommand{\cC}{\mathcal{C}}\newcommand{\cD}{\mathcal{D}}
\newcommand{\cE}{\mathcal{E}}\newcommand{\cF}{\mathcal{F}}
\newcommand{\cG}{\mathcal{G}}\newcommand{\cH}{\mathcal{H}}
\newcommand{\cI}{\mathcal{I}}\newcommand{\cJ}{\mathcal{J}}
\newcommand{\cK}{\mathcal{K}}\newcommand{\cL}{\mathcal{L}}
\newcommand{\cM}{\mathcal{M}}\newcommand{\cN}{\mathcal{N}}
\newcommand{\cO}{\mathcal{O}}\newcommand{\cP}{\mathcal{P}}
\newcommand{\cQ}{\mathcal{Q}}\newcommand{\cR}{\mathcal{R}}
\newcommand{\cS}{\mathcal{S}}\newcommand{\cT}{\mathcal{T}}
\newcommand{\cU}{\mathcal{U}}\newcommand{\cV}{\mathcal{V}}
\newcommand{\cW}{\mathcal{W}}\newcommand{\cX}{\mathcal{X}}
\newcommand{\cY}{\mathcal{Y}}\newcommand{\cZ}{\mathcal{Z}}

\newcommand{\bA}{\mathbb{A}}\newcommand{\bB}{\mathbb{B}}
\newcommand{\bC}{\mathbb{C}}\newcommand{\bD}{\mathbb{D}}
\newcommand{\bE}{\mathbb{E}}\newcommand{\bF}{\mathbb{F}}
\newcommand{\bG}{\mathbb{G}}\newcommand{\bH}{\mathbb{H}}
\newcommand{\bI}{\mathbb{I}}\newcommand{\bJ}{\mathbb{J}}
\newcommand{\bK}{\mathbb{K}}\newcommand{\bL}{\mathbb{L}}
\newcommand{\bM}{\mathbb{M}}\newcommand{\bN}{\mathbb{N}}
\newcommand{\bO}{\mathbb{O}}\newcommand{\bP}{\mathbb{P}}
\newcommand{\bQ}{\mathbb{Q}}\newcommand{\RR}{\mathbb{R}}
\newcommand{\bS}{\mathbb{S}}\newcommand{\bT}{\mathbb{T}}
\newcommand{\bU}{\mathbb{U}}\newcommand{\bV}{\mathbb{V}}
\newcommand{\bW}{\mathbb{W}}\newcommand{\bX}{\mathbb{X}}
\newcommand{\bY}{\mathbb{Y}}\newcommand{\bZ}{\mathbb{Z}}

\newcommand{\B}[1]{\mathbf{#1}}
\newcommand{\BB}[1]{\mathbb{#1}}





\definecolor{codegreen}{rgb}{0,0.6,0}
\definecolor{codegray}{rgb}{0.5,0.5,0.5}
\definecolor{codepurple}{rgb}{0.58,0,0.82}
\definecolor{backcolour}{rgb}{0.95,0.95,0.92}

\lstdefinestyle{mystyle}{
    backgroundcolor=\color{backcolour},   
    commentstyle=\color{codegreen},
    keywordstyle=\color{magenta},
    numberstyle=\tiny\color{codegray},
    stringstyle=\color{codepurple},
    basicstyle=\ttfamily\footnotesize,
    breakatwhitespace=false,         
    breaklines=true,                 
    captionpos=b,                    
    keepspaces=true,                 
    numbers=left,                    
    numbersep=5pt,                  
    showspaces=false,                
    showstringspaces=false,
    showtabs=false,                  
    tabsize=2
}

\lstset{style=mystyle}


\begin{document}

\section*{Exercise 4: Exercise 1.33 from the Course Text}
\begin{pro}[Proposition 1.29]
    Let $p$ be a prime and let $a$ be an integer not divisible by $p$. Suppose that
    $a^n \equiv 1 (\mod p)$. Then the order of $a\mod p$ divides $n$. In particular, 
    the order of $a$ divides $p-1$
\end{pro}

\begin{thm}(Primitive Root Theorem)
Let $p$ a prime number. Then there exists an element $g \in F^*_p$ whose powers give 
every element of $F^*_p$. i.e., $$F^*_p = \{1, g, g^2, g^3, \cdots, g^{p-2}\}.$$ Elements
with this property are called \emph{primitive roots} of $F_p$ or \emph{generators} of $F^*_p$. 
They are the elements $F^*_p$ having order $p-1$.
\end{thm}

\subsection*{Task}
Let $p$ be a prime and let $q$ be a prime that divides $p-1$.
\begin{enumerate}
    \item Let $a\in F^*_p$ and let $b=a^{(p-1)/q}.$ Prove that either $b=1$ or else $b$ has order $q$.
    (Recall that the order of $b$ is the smallest $k\ge 1$ such that $b^k = 1$ in $F^*_p$. \emph{Hint:}
    Use Proposition 1.29.)

    \item  Suppose that we want to find an element of $F^*_p$ of order $q$. Using (1), we can randomly
    choose a value of $a \in F^*_p$ and check whether $b = a^{(p-1)/q}$ satisfies $b\ne 1.$ How likely
    are we to succeed? in order words, compute the value of the ratio 
    $$\#\{a \in F^*_p : a^{(p-1)/q} \ne 1\}/ \# F^*_p.$$
    (\emph{Hint:} Use the Primitive Root Theorem) 
\end{enumerate}

\begin{proof}
    \begin{enumerate}
        \item Let $a\in F^*_p$ and let $b=a^{(p-1)/q}.$ Since $q$ is a prime that divides $p-1$, 
        then $p-1 = kq$ for some integer $k$.  This means the order of every element $a$ in $F^*_p$
        divides $p-1$. 

        Since $b=a^{(p-1)/q},$ we have $b^q = (a^{(p-1)/q})^q = a^{p-1}$. By Fermat's Little Theorem,
        we have $a^{p-1} = 1 \mod p.$ Thus $b^q = 1$. 

        Now the order of $b$ must be a divisor of $q$  by Proposition 1.29, since $b^q  = 1$. The 
        divisors of $q$ are $1$ and $q$. If the order is $1$, then $b$ must be 1 otherwise $q$ is 
        the order of $b$.

        \item The number of elements in $F^*_p$ with order $q$ is $\phi(q)$, where $\phi$ is the Euler
        totient function. But $\phi(q)=q-1.$ Also, the number of of elements in $F^*_p$ is $p-1$. 
        Therefore the likelihood of finding such an element is $\frac{q-1}{p-1}.$
    \end{enumerate}
\end{proof}

\section*{Exercise 5: Exercise 1.35 from the Course Text}
\begin{pro}
    Let $p$ be a prime such that $q = \frac{1}{2}(p-1)$ is also prime. Suppose that g is an integer
    satisfying 
    \begin{itemize}
        \item $g \not\equiv 0 (\mod p)$ 
        \item $g \not\equiv \pm (\mod p)$ 
        \item $g^q \not\equiv 1 (\mod p)$ 
    \end{itemize}
    Prove that $g$ is a pimitive root modulo p.
\end{pro}
\begin{proof}
    By the Primitive Root Theorem, there exists an element in $F^*_p$ whose powers give every
    element of $\mathbb{Z}^*_p$. By Proposition 0.1, the order of any $g \in \mathbb{Z}^*_p$
    must divide $p-1 = 2q$, which is the order of $\mathbb{Z}^*_p$. The possible factors of $2q$
    are $1, 2, q$ and $2q$.
    
    Since $g \not\equiv 0 (\mod p)$, then there is no zero elements in the
    multiplicative group. In addition, since $g \not\equiv \pm (\mod p)$, then $g$ cannot have 
    order 1 or 2. Finally, since $g^q \not\equiv 1 (\mod p)$, $g$ cannot have order $q$. 

    So, the only possible oder is $p-1 = 2q$. Thus, g is  of order $2q$, and hence a primitive
    root modulo $p$.

    
\end{proof}
\end{document}

